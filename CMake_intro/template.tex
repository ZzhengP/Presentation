



%\documentclass[8 pt]{beamer}
\documentclass[10 pt]{beamer}

\mode<presentation>
{
  \usetheme{Warsaw}
  \useoutertheme{infolines}
  \usecolortheme{ste}
  \setbeamercovered{transparent}
}


\usepackage{amsmath}
\usepackage{amssymb}
%\usepackage{stmaryrd}
\usepackage[english]{babel}
\usepackage[latin1]{inputenc}
\usepackage{times}
\usepackage{bm}
\usepackage[OT1]{fontenc}
%\usepackage{algorithmic}
\usepackage{verbatim}
\usepackage{tikz} % TikZ + Libraries
\usetikzlibrary{arrows,shapes,positioning,shadows,trees}
\usepackage{pgfplots}
\usepackage{psfrag}

%==================================================
% Custom Definitions

\newcommand{\RED}{\color[rgb]{0.90,0,0}}
\newcommand{\BLUE}{\color[rgb]{0,0,0.69}}
\newcommand{\GREEN}{\color[rgb]{0,0.69,0}}
\newcommand{\PURPLE}{\color[rgb]{0,0.69,0.69}}
\newcommand{\BLACK}{\color[rgb]{0,0,0}}

%\includeonlyframes{current}

\title[CMake short tuto] % (optional, use only with long paper titles)
{CMake: A very simple introduction}

\author[P. ZHENG] % (optional, use only with lots of authors)
{ {\bf Pu ZHENG} \\ PhD student, IMAG, Grenoble \\ {\it pu.zheng@univ-grenoble-alpes.fr}}

\institute[IMAG] % (optional, but mostly needed)
{IMAG}
% - Use the \inst command only if there are several affiliations.
% - Keep it simple, no one is interested in your street address.

\date[03/09/2019] % (optional, should be abbreviation of conference name)
{}
% - Either use conference name or its abbreviation.
% - Not really informative to the audience, more for people (including
%   yourself) who are reading the slides online

\subject{}
% This is only inserted into the PDF information catalog. Can be left
% out. 


% If you have a file called "university-logo-filename.xxx", where xxx
% is a graphic format that can be processed by latex or pdflatex,
% resp., then you can add a logo as follows:

%\logo{\includegraphics[height=0.2cm]{logo.pdf}}

% Delete this, if you do not want the table of contents to pop up at
% the beginning of each subsection:
 \AtBeginSection[]
 {
   \begin{frame}<beamer>{Outline}
     \tableofcontents[currentsection]
   \end{frame}
}
 \AtBeginSubsection[]
 {
   \begin{frame}<beamer>{Outline}
     \tableofcontents[currentsection,currentsubsection]
   \end{frame}
}


% If you wish to uncover everything in a step-wise fashion, uncomment
% the following command: 

%\beamerdefaultoverlayspecification{<+->}

%%%%%%%%%%%%%%%%%%%%<MISC>%%%%%%%%%%%%%%%%%%%%
% DEFINE NXP COLORS
\definecolor{nxpN}{RGB}{251,178,28}
\definecolor{nxpX}{RGB}{110,171,221}
\definecolor{nxpP}{RGB}{190,213,48}

%\includeonlyframes{current}

%%%%%%%%%%%%%%%%%%%%<START DOCUMENT>%%%%%%%%%%%%%%%%%%%%


\begin{document}

\begin{frame}
  \titlepage
\end{frame}

\begin{frame}{Outline}
  \tableofcontents
\end{frame}

\section{What is CMake ?}

\begin{frame}{What is CMake ?}

\begin{block}{CMake}
			CMake is a cross-platform, open-source build system. This tool will allow you to test, compile and create packages of your source code.
\end{block}

\pause

\begin{exampleblock}{Run CMake}
	In order to run a CMake projet we do:
\begin{itemize}
		\item mkdir build
		\item  cd build
		\item cmake ..
		\item make 
\end{itemize}
\end{exampleblock}

\end{frame}

\section{A case study}

\begin{frame}{A case study}
\tikzstyle{startstop} = [rectangle, rounded corners, minimum width=2cm, minimum height=0.8cm,text centered, draw=black, fill=red!30]
\tikzstyle{process} = [rectangle, minimum width=2cm, minimum height=0.8cm, text centered, draw=black, fill=orange!30]
\tikzstyle{file} = [rectangle, minimum width=2cm, minimum height=0.8cm, text centered, draw=black]
\tikzstyle{arrow} = [thick,->,>=stealth]
\begin{figure}
\centering
\begin{tikzpicture}[node distance=1.5cm]
%Firts level
\node (start) [startstop] {Robot Project};

%Second level 
\node (include) [startstop, below of = start] {include};
\node (src) [startstop, right of = include, xshift = 1cm] {src};
\node (example) [startstop, left of = include, xshift = -1cm] {example};
\node (CMakeList1) [file, left of = example, xshift = -1cm] {CMakeList.txt};
%Third level 
\node (rModelh) [process, below of = include, xshift = -1cm] {Robot model};
\node (rModelc) [process, below of = src, xshift = 1.2cm] {Robot model};
\node (loadModel) [file, below of = example, xshift = -1cm] {loadModel.cpp};
\node (CMakeList3) [file, left of = loadModel, xshift = -1cm] {CMakeList.txt};
\node (CMakeList2) [file, left of = rModelc, xshift = -0.8cm] {CMakeList.txt};
%Fourth level 
\node (modelh) [file, below of = rModelh] {model.h};
\node (modelcpp) [file, below of = rModelc] {model.cpp};

\draw [arrow] (start) -- (include);
\draw [arrow] (start) -- (src);
\draw [arrow] (start) -- (example);
\draw [arrow] (example) -- (loadModel);
\draw [arrow] (include) -- (rModelh);
\draw [arrow] (src) -- (rModelc);
\draw [arrow] (rModelc) -- (modelcpp);
\draw [arrow] (rModelh) -- (modelh);
\draw [arrow] (start) -- (CMakeList1);
\draw [arrow] (src) -- (CMakeList2);
\draw [arrow] (example) -- (CMakeList3);
\end{tikzpicture}
\caption{A classical robotic C++ project structure}
\end{figure}
	\end{frame}

\begin{frame}{Illustrative case}
\begin{block}{Project organization}
		The main project is called "robot project", in which we create sub-directory such as "example","include","src" in order to keep our codes modular. 
\end{block}

\pause
\begin{exampleblock}{CMakeList}
The CMakeList file is very important of several useful design patters for CMake. In this file we precise with "flags" to let cmake know how to build this project. 
\end{exampleblock}
\end{frame}

\subsection{CMakeList explication}

\begin{frame}{First CMakeList in "Robot Project"}
\begin{exampleblock}{Explication}
We explain the flags written in CMakeList in following: 
\end{exampleblock}
\begin{itemize}
	\item  cmake\_minimum\_require(VERSION 3.11)    \textit{Required version}
	\item  project(Robot6R)   				 \textit{Name of this project}
	\item find\_package(cartin REQUIRED COMPONENTS roscpp rospy ..... )  \textit{Find the external package used in this project }
	\item include\_directories(include \$ \{CATKIN\_INCLUDE\_DIRS\ ... )  \textit{Include the header of external package }
    \item ADD\_SUBDIRECTORY(examples src include) \textit{ add other sub-directory to project }
\end{itemize}
\end{frame}

\begin{frame}{CMakeList in "src"}
\begin{exampleblock}{Explication}
In src, we write our own library 
\end{exampleblock}
\begin{itemize}
	\item  set(HEADER\_LIST "give the path to the project include directory")   \textit{In order to include our header to the src}
	\item  add\_linbrary(robot6R\_library)  				 \textit{Name of this package or library}
	\item find\_package(cartin REQUIRED COMPONENTS roscpp rospy ..... )  \textit{Find the external package used in this project }
	\item include\_directories(include \$ \{CATKIN\_INCLUDE\_DIRS\ ... )  \textit{Include the header of external package }
	\item ADD\_SUBDIRECTORY(examples src include) \textit{ add other sub-directory to project }
\end{itemize}
\end{frame}
\end{document}

%%%%%%%%%%%%%%%%%%%%%% End
